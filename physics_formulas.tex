\documentclass{article}
\usepackage{amsmath}
\usepackage{amssymb}
\usepackage{geometry}
\geometry{a4paper, margin=1in}
\title{Fysische Formules - Volledig Document}
\author{}
\date{}

\begin{document}

\maketitle

\section*{Kinetische energie}
\[ \text{De kinetische energie (KE) van een object met massa } m \text{ en snelheid } v \text{ wordt gegeven door:} \\
\]
\[ KE = \frac{1}{2} mv^2 \quad \text{Dit is de energie die een object bezit door zijn beweging.} \\
\]

\section*{Potentiële energie}
\[ \text{De potentiële energie (PE) van een object op een hoogte } h \text{ boven een referentiepunt, met massa } m \text{ en gravitatieversnelling } g, \text{ wordt gegeven door:} \\
\]
\[ PE = mgh \quad \text{Dit is de energie die een object bezit door zijn positie in een zwaartekrachtveld.} \\
\]

\section*{Eindsnelheid (constante versnelling)}
\[ \text{Bij een constante versnelling } a \text{ wordt de snelheid } v \text{ na tijd } t \text{ gegeven door:} \\
\]
\[ v = v_0 + at \quad \text{Hier is } v_0 \text{ de beginsnelheid van het object.} \\
\]

\section*{Afgelegde afstand (constante versnelling)}
\[ \text{De afstand } s \text{ afgelegd door een object met beginsnelheid } v_0 \text{ en constante versnelling } a \text{ na tijd } t \text{ wordt gegeven door:} \\
\]
\[ s = v_0 t + \frac{1}{2} at^2 \\
\]

\section*{Snelheid gerelateerd aan afgelegde afstand}
\[ \text{De relatie tussen de snelheid } v \text{ en de afgelegde afstand } s \text{ bij constante versnelling } a \text{ wordt gegeven door:} \\
\]
\[ v^2 = v_0^2 + 2as \\
\]

\section*{Tweede wet van Newton}
\[ \text{De kracht } F \text{ die op een object met massa } m \text{ werkt, wordt gegeven door:} \\
\]
\[ F = ma \quad \text{Hier is } a \text{ de versnelling van het object.} \\
\]

\section*{Gewichtskracht}
\[ \text{De gewichtskracht } F_{\text{zwaartekracht}} \text{ op een object met massa } m \text{ in een gravitatieveld met versnelling } g \text{ wordt gegeven door:} \\
\]
\[ F_{\text{zwaartekracht}} = mg \\
\]

\section*{Wrijvingskracht}
\[ \text{De wrijvingskracht } F_{\text{wrijv}} \text{ wordt bepaald door:} \\
\]
\[ F_{\text{wrijv}} = \mu N \quad \text{Hier is } \mu \text{ de wrijvingscoëfficiënt en } N \text{ de normaalkracht.} \\
\]

\section*{Impuls}
\[ \text{De impuls } p \text{ van een object met massa } m \text{ en snelheid } v \text{ wordt gegeven door:} \\
\]
\[ p = mv \\
\]

\section*{Behoud van impuls}
\[ \text{Bij een botsing blijft de totale impuls behouden:} \\
\]
\[ m_1 v_1 + m_2 v_2 = (m_1 + m_2) v_f \quad \text{Dit geldt voor inelastische botsingen.} \\
\]

\section*{Wet van behoud van energie}
\[ \text{De totale mechanische energie (kinetische en potentiële) blijft constant in een gesloten systeem:} \\
\]
\[ KE + PE = \text{constant} \\
\]

\section*{Arbeid}
\[ \text{De arbeid } W \text{ verricht door een kracht } F \text{ over een afstand } d \text{ wordt gegeven door:} \\
\]
\[ W = Fd \cos(\theta) \quad \text{Hier is } \theta \text{ de hoek tussen kracht en verplaatsing.} \\
\]

\section*{Vermogen}
\[ \text{Het vermogen } P \text{ is de hoeveelheid arbeid } W \text{ per tijdseenheid } t: \\
\]
\[ P = \frac{W}{t} \\
\]

\section*{Moment van een kracht}
\[ \text{Het moment } \tau \text{ van een kracht } F \text{ met arm } r \text{ wordt gegeven door:} \\
\]
\[ \tau = r \times F \sin(\theta) \\
\]

\section*{Traagheidsmoment}
\[ \text{Het traagheidsmoment } I \text{ voor een systeem van massa's wordt gegeven door:} \\
\]
\[ I = \sum m_i r_i^2 \\
\]

\section*{Hoekversnelling}
\[ \text{De hoekversnelling } \alpha \text{ is de verhouding van het moment } \tau \text{ tot het traagheidsmoment } I: \\
\]
\[ \alpha = \frac{\tau}{I} \\
\]

\section*{Kinetische energie van rotatie}
\[ \text{De kinetische energie van een draaiend object met traagheidsmoment } I \text{ en hoeksnelheid } \omega \text{ wordt gegeven door:} \\
\]
\[ KE_{\text{rotatie}} = \frac{1}{2} I \omega^2 \\
\]

\section*{Periode van een slinger}
\[ \text{De periode } T \text{ van een slinger met lengte } L \text{ in een gravitatieveld } g \text{ wordt gegeven door:} \\
\]
\[ T = 2 \pi \sqrt{\frac{L}{g}} \\
\]

\section*{Golfsnelheid}
\[ \text{De golfsnelheid } v \text{ wordt berekend als het product van de golflengte } \lambda \text{ en frequentie } f: \\
\]
\[ v = \lambda f \\
\]

\section*{Energie in een trillend systeem}
\[ \text{De energie } E_{\text{golf}} \text{ in een trillend systeem met amplitude } A \text{ en veerconstante } k \text{ wordt gegeven door:} \\
\]
\[ E_{\text{golf}} = \frac{1}{2} k A^2 \\
\]

\section*{Wet van Ohm}
\[ \text{De spanning } V \text{ over een geleider met weerstand } R \text{ en stroomsterkte } I \text{ wordt gegeven door:} \\
\]
\[ V = IR \\
\]

\section*{Elektrisch vermogen}
\[ \text{Het elektrisch vermogen } P \text{ is het product van spanning } V \text{ en stroomsterkte } I: \\
\]
\[ P = IV \\
\]

\section*{Kracht op een lading}
\[ \text{De kracht } F \text{ op een lading } q \text{ in een elektrisch veld } E \text{ wordt gegeven door:} \\
\]
\[ F = qE \\
\]

\section*{Magnetische flux}
\[ \text{De magnetische flux } \Phi \text{ door een oppervlak met magnetisch veld } B \text{ en oppervlakte } A \text{ wordt gegeven door:} \\
\]
\[ \Phi = B \cdot A \cos(\theta) \\
\]

\section*{Kracht op een bewegende lading}
\[ \text{De kracht } F \text{ op een bewegende lading } q \text{ in een magnetisch veld } B \text{ wordt gegeven door:} \\
\]
\[ F = Bqv \sin(\theta) \\
\]

\section*{Goniometrische basisverhouding}
\[ \text{De fundamentele goniometrische identiteit is:} \\
\]
\[ \sin^2 \theta + \cos^2 \theta = 1 \\
\]

\section*{Secant en tangens relatie}
\[ \text{Een belangrijke relatie tussen tangens en secant is:} \\
\]
\[ 1 + \tan^2 \theta = \sec^2 \theta \\
\]

\section*{Definities van de goniometrische functies}
\[ \sin \theta = \frac{\text{overstaande zijde}}{\text{schuine zijde}} \quad \cos \theta = \frac{\text{aanliggende zijde}}{\text{schuine zijde}} \quad \tan \theta = \frac{\sin \theta}{\cos \theta} \\
\]
\[ \cot \theta = \frac{1}{\tan \theta} \quad \sec \theta = \frac{1}{\cos \theta} \quad \csc \theta = \frac{1}{\sin \theta} \\
\]

\section*{Stelling van Pythagoras}
\[ \text{In een rechthoekige driehoek geldt:} \\
\]
\[ a^2 + b^2 = c^2 \quad \text{Waarbij } c \text{ de schuine zijde is.} \\
\]

\section*{Goniometrische som- en verschilformules}
\[ \text{De som- en verschilformules zijn:} \\
\]
\[ \sin(a \pm b) = \sin a \cos b \pm \cos a \sin b \\
\]
\[ \cos(a \pm b) = \cos a \cos b \mp \sin a \sin b \\
\]
\[ \tan(a \pm b) = \frac{\tan a \pm \tan b}{1 \mp \tan a \tan b} \\
\]

\section*{Dubbele hoekformules}
\[ \sin(2\theta) = 2 \sin \theta \cos \theta \quad \cos(2\theta) = \cos^2 \theta - \sin^2 \theta \\
\]
\[ \tan(2\theta) = \frac{2 \tan \theta}{1 - \tan^2 \theta} \\
\]

\section*{Halve hoekformules}
\[ \sin^2 \theta = \frac{1 - \cos(2\theta)}{2} \quad \cos^2 \theta = \frac{1 + \cos(2\theta)}{2} \\
\]
\[ \tan^2 \theta = \frac{1 - \cos(2\theta)}{1 + \cos(2\theta)} \\
\]

\end{document}
